\section{CashFlows}



\subsection{Global}
  Exactly one \xmlNode{Global} block has to be provided. The \xmlNode{Global} block does not have
  any attributes.


  The \xmlNode{Global} node recognizes the following subnodes:
  \begin{itemize}
    \item \xmlNode{Indicator}: \xmlDesc{comma-separated strings}, 
      List of cash flows considered in the computation of the economic indicator. See later for the
      definition           of the cash flows. Only cash flows listed here are considered, additional
      cash flows defined, but not listed are ignored.
      The \xmlNode{Indicator} node recognizes the following parameters:
        \begin{itemize}
          \item \xmlAttr{name}: \xmlDesc{comma-separated strings, required}, 
             The names of the economic indicators that should be computed. So far, \textbf{'NPV'},
            \textbf{'NPV\_search'}, \textbf{'IRR'} and \textbf{'PI'} are supported. More than one
            indicator can be asked for. The \xmlAttr{name} attribute can contain a comma-separated
            list as shown in the example in Listing  ref{lst:InputExample}. \\  \textbf{Note on IRR
            and PI search}: It should be noted that although the only search keyword allowed in
            \xmlAttr{name} is \textbf{NPV\_search}, it is possible to perform IRR and PI searches as
            well. \begin{itemize} \item To do an IRR search, the DiscountRate is set to the desired
            IRR and a NPV search with the target of ‘0’ is performed. \item To perform a PI search,
            an NPV search can be performed where the target PI is multiplied with the initial
            investment. \end{itemize}
          \item \xmlAttr{target}: \xmlDesc{float, optional}, 
            Target value for the NPV search, i.e. \textbf{'0'} will look for ‘$x$’ so that $NPV(x) =
            0$.
      \end{itemize}

      The \xmlNode{Indicator} node recognizes the following subnodes:
      \begin{itemize}
        \item \xmlNode{DiscountRate}: \xmlDesc{float}, 
          The discount rate used to compute the NPV and PI. Not used for the computation of the IRR
          (although it must be input).

        \item \xmlNode{tax}: \xmlDesc{float}, 
          The standard tax rate used to compute the taxes if no other tax rate is specified in the
          componet blocks. This is a required input. If a tax rate is specified inside a component
          block, the componet will use that tax rate. If no tax rate is specified in a component,
          this standard tax rate is used for the component. See later in the definition of the cash
          flows for more details how the tax rate is used.

        \item \xmlNode{inflation}: \xmlDesc{float}, 
          The standard inflation rate used to compute the inflation if no other inflation rate is
          specified in the componet blocks. This is a required input. If a inflation rate is
          specified inside a component block, the componet will use that inflation rate. If no
          inflation rate is specified in a component, this standard inflation rate is used for the
          component. See later in the definition of the cash flows

        \item \xmlNode{ProjectTime}: \xmlDesc{integer}, 
          This is a optional input. If it is included in the input, the global project time is not
          the LCM of all components (see \xmlNode{Indicator} attribute \xmlAttr{name} for more
          information), but the time indicated here.
      \end{itemize}
  \end{itemize}




\subsection{Component}
  The user can define as many \xmlNode{Component} blocks as needed. A component is typically a part
  of the system that has the same lifetime and
  the same cash flows, i.e. for example a gas turbine, a battery or a nuclear plant. Each component
  needs to have a \xmlAttr{name} attribute that is unique.
  Each \xmlNode{Component} has to have one \xmlNode{Life\_time} block and as many \xmlNode{CashFlow}
  blocks as needed.

  The \xmlNode{Component} node recognizes the following parameters:
    \begin{itemize}
      \item \xmlAttr{name}: \xmlDesc{string, required}, 
        The unique name of the component.
  \end{itemize}

  The \xmlNode{Component} node recognizes the following subnodes:
  \begin{itemize}
    \item \xmlNode{Life\_time}: \xmlDesc{integer}, 
      The lifetime of the component in years. This is used to compute the least common multiple
      (LCM) of all components involved in the                                 computation of the
      economics indicator. For more details see NPV, IRR and PI explanations above.

    \item \xmlNode{StartTime}: \xmlDesc{integer}, 
      This is a optional input. If this input is specified for one or more components, the
      \xmlNode{Global}                                 input \xmlNode{ProjectTime} is required. This
      input specifies the year in which this component is going to be build for the first time,
      i.e. is going to be included in the cash flows. The default is 0 and the componet is build at
      the start of the project, i.e. at project year 0.                                 For example,
      if the \xmlNode{ProjectTime} is 100 years, and for this component, the \xmlNode{StartTime} is
      20 years, the cash flows for this                                 component are going to be
      zero for years 0 to 19 of the project. Year 20 of the project will be year 0 of this component
      and so on                                 (project year 21 will be component year 1 etc.).

    \item \xmlNode{Repetitions}: \xmlDesc{integer}, 
      This is a optional input. If this input is specified for one or more components, the
      \xmlNode{Global}                                 input \xmlNode{ProjectTime} is required. This
      input specifies the number of times this component is going to be rebuilt. The default is 0,
      which indicates that the component is going to be rebuild indefinitely until the project end
      (\xmlNode{ProjectTime}) is reached.                                 Lets assume the
      \xmlNode{ProjectTime} is 100 years, and the component \xmlNode{Life\_time} is 20 years.
      Specifying 3 repetitions of this                                 component will build 3
      components in succession, at years 0, 20 and 40. For years 61 to 100 of the project, the cash
      flows for this component will be zero.

    \item \xmlNode{tax}: \xmlDesc{float}, 
      This is a optional input. If the tax rate is specified here, i.e. inside the component block,
      the componet will use this tax rate.                                 If no tax rate is
      specified in the component, the standard tax rate from the \xmlNode{Global} block is used for
      the component.

    \item \xmlNode{inflation}: \xmlDesc{float}, 
      This is a optional input. If the inflation rate is specified here, i.e. inside the component
      block,                                 the componet will use this inflation rate. If no
      inflation rate is specified in the component, the standard inflation rate from the
      \xmlNode{Global}                                 block is used for the component.

    \item \xmlNode{CashFlows}:
      The user can define any number of 'cash flows' for a component. Each cash flow is of the form
      given in                                   Eq. \ref{eq:CF} where $y$ is the year from 0
      (capital investment) to the end of the \xmlNode{Life\_time} of the component.
      \begin{equation}\label{eq:CF}
      CF\_{y}=mult\cdot\alpha\_{y}\left ( \frac{driver\_{y}}{ref} \right )^{X}
      \end{equation}

      The \xmlNode{CashFlows} node recognizes the following subnodes:
      \begin{itemize}
        \item \xmlNode{Capex}:
          The cash flow for capital expenditures
          The \xmlNode{Capex} node recognizes the following parameters:
            \begin{itemize}
              \item \xmlAttr{name}: \xmlDesc{string, required}, 
                The name of the Cash flow. Has to be unique across all components. This is the name
                that can be listed in the                             \xmlNode{Indicator} node of
                the \xmlNode{Global} block.
              \item \xmlAttr{tax}: \xmlDesc{[True, Yes, 1, False, No, 0, t, y, 1, f, n, 0], optional}, 
                Can be \textbf{true} or \textbf{false}. If it is \textbf{true}, the cash flow is
                multiplied by $(1-tax)$, where tax                                 is the tax rate
                given in \xmlNode{tax} in the \xmlNode{Global}
                block. As an example, the cash flow of \textit{comp2} for year 119 in Listing
                \ref{lst:InputExample} would become $CF^{comp2}\_{39}(1-tax)$.
                If a cash flow with \xmlAttr{tax}$=$\textbf{true} is the driver of another cash
                flow, the cash flow without the tax is used as driver for the new cash flow.
                The limitation of having a global tax rate will be lifted in future version of the
                \textbf{TEAL.CashFlow} module. It is planned to have the possibility to
                input different tax rates for each component, since they might be in different tax
                regions.
              \item \xmlAttr{inflation}: \xmlDesc{[real, none], optional}, 
                Can be \textbf{real, nominal} or \textbf{none}. If it is \textbf{real}, the cash
                flow is multiplied by                               $(1+inflation)^{-y}$. If it is
                \textbf{nominal}, the cash flow is multiplied by $(1+inflation)^y$.
                In both cases, inflation is given by \xmlNode{inflation} in the \xmlNode{Global}
                block. Furthermore, $y$ goes from year 0 (capital investment)
                to the LCM of all component lifetimes.                               This means that
                the cash flows as expressed in Listing \ref{lst:InputExample} are multiplied with
                the infloation seen from today, i.e. the cash                               flow for
                \textit{comp2} for year 119 assuming it includes \textbf{real}
                inflation would be $CF^{comp2}\_{39}(1+inflation)^{-119}$
                If a cash flow with \xmlAttr{inflation} equal \textbf{real} or \textbf{nominal} is
                the driver of another cash flow, the cash flow without
                the inflation is used as driver for the new cash flow.
              \item \xmlAttr{mult\_target}: \xmlDesc{[True, Yes, 1, False, No, 0, t, y, 1, f, n, 0], optional}, 
                Can be \textbf{true} or \textbf{false}. If \textbf{true}, it means that this cash
                flow multiplies                               the search variable '$x$' as explained
                in the NPV\_search option above.                               If the NPV\_search
                option is used, al least one cash flow has to have
                \xmlAttr{mult\_target}$=$\textbf{true}.
              \item \xmlAttr{multiply}: \xmlDesc{string, optional}, 
                This is an optional attribute. This can be the name of any scalar variable passed in
                from RAVEN. This number                                 is $mult$ in Eq. \ref{eq:CF}
                that multiplies the cash flow.
          \end{itemize}

          The \xmlNode{Capex} node recognizes the following subnodes:
          \begin{itemize}
            \item \xmlNode{driver}: \xmlDesc{comma-separated strings, integers, and floats}, 
              The $driver$ in Eq. \ref{eq:CF} of the cash flow. This can be any variable passed in
              from RAVEN or the name                               of another cash flow. If it is
              passed in from RAVEN, it has to be either a scalar or a vector with length
              \xmlNode{Life\_time} + 1.                               If its a scalar, all
              $driver\_{y}$ in Eq. \ref{eq:CF}  are the same for all years of the project life. If it
              is a vector instead, each                               year of the project
              \xmlNode{Life\_time} will have its corresponding value for the driver. If the driver
              is another                               cash flow, the project \xmlNode{Life\_time}
              of the component to which the driving cash flow belongs has to be the same than the
              project

            \item \xmlNode{alpha}: \xmlDesc{comma-separated strings, integers, and floats}, 
              $\alpha\_{y}$ multiplier of the cash flow (see Eq. \ref{eq:CF}). Similar to
              \xmlNode{driver}, can be                               either scalar or vector. If a
              vector, exactly \xmlNode{Life\_time}$ + 1$                               values are
              expected. One for $y=0$ to $y=$\xmlNode{Life\_time}. If a scalar, we assume alpha is
              zero for all years of the lifetime                               of the component
              except the year zero (the provided scalar value will be used for year zero), which is
              the construction year.

            \item \xmlNode{reference}: \xmlDesc{float}, 
              The $ref$ value of the cash flow (see Eq. \ref{eq:CF}).

            \item \xmlNode{X}: \xmlDesc{float}, 
              The $X$ exponent (economy of scale factor) of the cash flow (see Eq. \ref{eq:CF}).

            \item \xmlNode{depreciation}: \xmlDesc{comma-separated strings, integers, and floats}, 
              INSERT
              The \xmlNode{depreciation} node recognizes the following parameters:
                \begin{itemize}
                  \item \xmlAttr{scheme}: \xmlDesc{[MACRS, custom], required}, 
                    -- no description yet --
              \end{itemize}
          \end{itemize}
      \end{itemize}

    \item \xmlNode{Recurring}:
      The cash flow for recurring cost, such as operation and maintenance cost.
      The \xmlNode{Recurring} node recognizes the following parameters:
        \begin{itemize}
          \item \xmlAttr{name}: \xmlDesc{string, required}, 
            The name of the Cash flow. Has to be unique across all components. This is the name that
            can be listed in the                             \xmlNode{Indicator} node of the
            \xmlNode{Global} block.
          \item \xmlAttr{tax}: \xmlDesc{[True, Yes, 1, False, No, 0, t, y, 1, f, n, 0], optional}, 
            Can be \textbf{true} or \textbf{false}. If it is \textbf{true}, the cash flow is
            multiplied by $(1-tax)$, where tax                                 is the tax rate given
            in \xmlNode{tax} in the \xmlNode{Global}                                 block. As an
            example, the cash flow of \textit{comp2} for year 119 in Listing \ref{lst:InputExample}
            would become $CF^{comp2}\_{39}(1-tax)$.                                 If a cash flow
            with \xmlAttr{tax}$=$\textbf{true} is the driver of another cash flow, the cash flow
            without the tax is used as driver for the new cash flow.
            The limitation of having a global tax rate will be lifted in future version of the
            \textbf{TEAL.CashFlow} module. It is planned to have the possibility to
            input different tax rates for each component, since they might be in different tax
            regions.
          \item \xmlAttr{inflation}: \xmlDesc{[real, none], optional}, 
            Can be \textbf{real, nominal} or \textbf{none}. If it is \textbf{real}, the cash flow is
            multiplied by                               $(1+inflation)^{-y}$. If it is
            \textbf{nominal}, the cash flow is multiplied by $(1+inflation)^y$.
            In both cases, inflation is given by \xmlNode{inflation} in the \xmlNode{Global} block.
            Furthermore, $y$ goes from year 0 (capital investment)                               to
            the LCM of all component lifetimes.                               This means that the
            cash flows as expressed in Listing \ref{lst:InputExample} are multiplied with the
            infloation seen from today, i.e. the cash                               flow for
            \textit{comp2} for year 119 assuming it includes \textbf{real}
            inflation would be $CF^{comp2}\_{39}(1+inflation)^{-119}$
            If a cash flow with \xmlAttr{inflation} equal \textbf{real} or \textbf{nominal} is the
            driver of another cash flow, the cash flow without                               the
            inflation is used as driver for the new cash flow.
          \item \xmlAttr{mult\_target}: \xmlDesc{[True, Yes, 1, False, No, 0, t, y, 1, f, n, 0], optional}, 
            Can be \textbf{true} or \textbf{false}. If \textbf{true}, it means that this cash flow
            multiplies                               the search variable '$x$' as explained in the
            NPV\_search option above.                               If the NPV\_search option is
            used, al least one cash flow has to have \xmlAttr{mult\_target}$=$\textbf{true}.
          \item \xmlAttr{multiply}: \xmlDesc{string, optional}, 
            This is an optional attribute. This can be the name of any scalar variable passed in
            from RAVEN. This number                                 is $mult$ in Eq. \ref{eq:CF}
            that multiplies the cash flow.
      \end{itemize}

      The \xmlNode{Recurring} node recognizes the following subnodes:
      \begin{itemize}
        \item \xmlNode{driver}: \xmlDesc{comma-separated strings, integers, and floats}, 
          The $driver$ in Eq. \ref{eq:CF} of the cash flow. This can be any variable passed in from
          RAVEN or the name                               of another cash flow. If it is passed in
          from RAVEN, it has to be either a scalar or a vector with length \xmlNode{Life\_time} + 1.
          If its a scalar, all $driver\_{y}$ in Eq. \ref{eq:CF}  are the same for all years of the
          project life. If it is a vector instead, each                               year of the
          project \xmlNode{Life\_time} will have its corresponding value for the driver. If the
          driver is another                               cash flow, the project
          \xmlNode{Life\_time} of the component to which the driving cash flow belongs has to be the
          same than the project

        \item \xmlNode{alpha}: \xmlDesc{comma-separated strings, integers, and floats}, 
          $\alpha\_{y}$ multiplier of the cash flow (see Eq. \ref{eq:CF}). Similar to
          \xmlNode{driver}, can be                               either scalar or vector. If a
          vector, exactly \xmlNode{Life\_time}$ + 1$                               values are
          expected. One for $y=0$ to $y=$\xmlNode{Life\_time}. If a scalar, we assume alpha is zero
          for all years of the lifetime                               of the component except the
          year zero (the provided scalar value will be used for year zero), which is the
          construction year.
      \end{itemize}
  \end{itemize}


